\documentclass[12pt]{article}
\usepackage[french]{babel}
\usepackage[utf8]{inputenc}
\usepackage{amsmath}
\usepackage{graphicx}
\usepackage[colorinlistoftodos]{todonotes}
\usepackage{float}
\usepackage{graphicx}
\usepackage{caption}
\usepackage{subcaption}
\usepackage{adjustbox}
\usepackage[euler]{textgreek}
\usepackage[left=1in,right=1in,top=1in,bottom=1in]{geometry}
%\usepackage{subfigure}
\usepackage{comment}
\usepackage{caption}
\usepackage{lastpage}
\usepackage[colorlinks,pdfpagelabels,pdfstartview = FitH,bookmarksopen = true,bookmarksnumbered = true,linkcolor = black,plainpages = false,hypertexnames = true,citecolor = black,pagebackref = true,urlcolor = black] {hyperref}
\usepackage{setspace}
\usepackage{silence}
\WarningFilter{latex}{Text page}
\usepackage{parskip}
\newcommand{\cms}{~cm\textsuperscript{-2}s\textsuperscript{-1} }

\graphicspath{{figures/}}   
% \renewcommand{\figurename}{Fig.}
\addto\captionsenglish{\renewcommand{\figurename}{Fig.}}

\begin{document}

\begin{titlepage}

\newcommand{\HRule}{\rule{\linewidth}{0.5mm}} % Defines a new command for the horizontal lines, change thickness here

\center % Center everything on the page
 
%----------------------------------------------------------------------------------------
%	HEADING SECTIONS
%----------------------------------------------------------------------------------------

\textsc{\LARGE Universit\'e Claude Bernard Lyon I}\\[1.5cm] % Name of your university/college
\textsc{\Large Master Data Science}\\[0.5cm] % Major heading such as course name
\textsc{\large Projet Data Mining}\\[0.5cm] % Minor heading such as course title

%----------------------------------------------------------------------------------------
%	TITLE SECTION
%----------------------------------------------------------------------------------------

\HRule \\[0.4cm]
{ \huge \bfseries D\'etection d'\'ev\'enements sur Twitter}\\[0.4cm] % Title of your document
\HRule \\[1.5cm]
 
%----------------------------------------------------------------------------------------
%	AUTHOR SECTION
%----------------------------------------------------------------------------------------

\begin{minipage}{0.4\textwidth}
\begin{flushleft} \large
\emph{Auteurs:}\\
Gregory \textsc{Howard} \textit{11207726} \\ 
Marine \textsc{Ruiz} \textit{11208141} \\ 
Jules \textsc{Sauvinet} \textit{p1412086}
\end{flushleft}
\end{minipage}
~
\begin{minipage}{0.4\textwidth}
\begin{flushright} \large
\emph{Professeur:} \\
Marc \textsc{Plantevit} % Supervisor's Name
\end{flushright}
\end{minipage}\\[2cm]

% If you don't want a supervisor, uncomment the two lines below and remove the section above
%\Large \emph{Author:}\\
%John \textsc{Smith}\\[3cm] % Your name

%----------------------------------------------------------------------------------------
%	DATE SECTION
%----------------------------------------------------------------------------------------

{\large \today}\\[1cm] % Date, change the \today to a set date if you want to be precise

%----------------------------------------------------------------------------------------
%	LOGO SECTION
%----------------------------------------------------------------------------------------

\includegraphics[height=5cm]{ucbl}\\[1cm] % Include a department/university logo - this will require the graphicx package
 
%----------------------------------------------------------------------------------------

\vfill % Fill the rest of the page with whitespace

\end{titlepage}

\begin{abstract}
L'objectif est de d\'etecter des \'ev\`enements \`a partir de tweets en exploitant des informations spatio-temporelles et textuelles.
Notre approche utilise la d\'etection d'\'ev\'enements multi-\'echelles dans les r\'eseaux sociaux introduite dans l'article de recherche... et d\'evelopp\'e par ... . Nous avons adapt\'e cette version \`a nos donn\'ees et nous avons ajout\'e des m\'ecanismes permettant le filtrage de robots.
\end{abstract}

%\begin{spacing}{1.15}
\pdfbookmark[1]{Contents}{toc}
\small{
\tableofcontents 
}


\newpage
% \listoffigures
% \newpage
% \listoftables 
%\end{spacing}


\section{Description de nos donn\'ees}
\label{sec:desc_donnees}

\textit{Source du fichier}

\paragraph{}
Nous avons un fichier de tweets r\'ecup\'er\'es \`a partir de Twitter. Nous avons les informations
suivantes sur chaque tweet :
\begin{itemize}
	\item l'identifiant du tweet
	\item la date et l'heure \`a laquelle le tweet a \'et\'e post\'e
	\item la position de la personne quand le tweet a \'et\'e post\'e (en latitude et longitude)
	\item identifiant du district (en fonction de la position)
	\item l'ensemble des r\'ef\'erences et hashtags du tweet
	\item l'identifiant de la personne qui a post\'e le tweet
	\item le pseudo de la personne qui a post\'e le tweet
	\item le nom complet de la personne qui a post\'e le tweet
	\item l'identifiant du lieu de cr\'eation du compte de la personne qui a post\'e le tweet
	\item certifi\'e si le compte est officiel ou si son identit\'e a \'et\'e prouv\'ee
	\item nombre de personnes qui suivent la personne qui a post\'e le tweet
	\item nombre d'amis de la personne qui a post\'e le tweet
	\item nombre de tweet d\'ej\`a post\'e par la personne qui a post\'e le tweet
	\item lieu du tweet au moment o\`u il a \'et\'e post\'e
\end{itemize}

Nous nous sommes servi de l'identifiant du tweet, du nom de la personne qui a post\'e le
tweet, des hashtags, de la date et de la position d'envoi du tweet.


\section{L'objectif}

\paragraph{}
Nous avons pour objectif de d\'etecter des \'ev\'enements journaliers \`a l'aide des tweets de
notre fichier de donn\'ees. Nous allons test\'e deux algorithmes diff\'erents puis comparer les
r\'esultats afin de tester les performances de chacun et de garder le meilleur.

\textit{A d\'etailler}

\newpage

\section{L'approche}

\paragraph{}
Pour notre premier algorithme, nous faisons des sous-ensembles de tweets en d\'ecoupant
nos donn\'ees par date. On a ainsi un ensemble de tweets par jour. On applique sur chaque
ensemble journalier de tweets, l'algorithme de clustering qui calcule la matrice de similarit\'e
et construit les clusters.

\paragraph{}
Pour notre second algorithme, nous supprimons avant tout traitement, les comptes robots
que nous sommes susceptibles d'avoir dans nos donn\'ees. Il est fort probable qu'ils
faussent notre clustering.

\paragraph{}
De plus, chacun de nos \'ev\'enements est d\'ecrit par 20 hashtags pertinents. Ces hashtags
pertinents sont d\'efinis comme \'etant les plus r\'ecurrents dans les tweets contenus dans
l'\'ev\'enement. Cependant, certains hashtags sont pr\'esents dans la description de
beaucoup d'autres \'ev\'enements, ce qui les rend moins pertinents. Nous voulons les
d\'etecter et ne pas les prendre en compte lors du clustering afin d'avoir une vraie description
de l'\'ev\'enement.

\newpage

\section{Description des \'el\'ements utiles \`a la d\'etection d'\'ev\'enements}

\textbf{Le tweet}
\newline
On d\'ecrit un tweet par : 

\begin{itemize}
\item son identifiant
\item le nom du Tweetos
\item les hashtags
\item la date et l'heure \`a laquelle le Tweetos a tweet\'e
\item sa position en latitude/longitude au moment du tweet
\item La date
\end{itemize}

On d\'ecoupe nos donn\'ees par date c'est \`a dire par journ\'ee.
Une date est caract\'eris\'ee par une journ\'ee, un mois et une ann\'ee.

\textbf{La matrice de similarit\'e}
\newline
La matrice de similarit\'e est une matrice triangulaire sup\'erieure de taille (nombre de
tweets) * (nombre de tweets).
D\'ecrire construction
Citer l'article en parlant rapidement de la compression de Haar

\textbf{Le clustering}
\newline
Le clustering se fait \`a partir de la matrice de similarit\'e.
D\'ecrire mieux l'algo
\newline
Citer le jar et expliquer rapidement ce qu'on maximise / minimise dans l'algorithme

\textbf{L'\'ev\'enement}
\newline
On d\'ecrit un \'ev\'enement par :
\begin{itemize}
\item une heure de d\'ebut et une heure de fin qui permettent de calculer une dur\'ee
\item une position centrale (longitude, latitude) calcul\'ee \`a partir de la moyenne de toutes
les positions (moyenne des longitudes, moyenne des longitudes)
\item le nombre de personnes diff\'erentes ayant post\'e des tweets
\item la liste des 20 hashtags les plus importants permettant de le d\'ecrire
\end{itemize}


\section{Algorithme de la d\'etection d'\'ev\'enements}

\paragraph{}
on utilise Knime pour filtrer les robots \'evidents --> ceux qui tweet beaucoup trop par
rapport aux autres
\newline
on traite ensuite tous les tweets d'une m\^eme journ\'ee
\newline
on enl\`eve ceux qui sont r\'egi par une loi G\'eom\'etrique (a voir)
\newline
on fait du clustering sur les tweets et on renvoie les clusters pertinents sous forme
d'\'ev\'enements

\section{Les options sur le clustering}

\textbf{L'\'elasticit\'e}

\paragraph{}
On suppose qu'un hashtag est fr\'equent pour un ensemble de tweets donn\'e s'il apparait
plus de 20 fois.
\newline
Ce nombre est statique, ce qui ne permettrait pas de d\'etecter de petits \'ev\'enements dans
une journ\'ee o\`u il y a eu peu de tweets. A l'inverse, on d\'etecte beaucoup d'\'ev\'enements sur
une journ\'ee o\`u il y a eu beaucoup de tweets.
Cela peut \^etre int\'eressant, mais il est parfois plus judicieux de d'adapter au nombre de
tweets post\'es dans une m\^eme journ\'ee. Ainsi, on d\'efinit un hashtag fr\'equent s'il apparait
dans plus de 20% des tweets.
Pour coder cette option nous avons d\'efinit un bool\'een, que l'on met \`a Vrai pour activer
l'\'elasticit\'e et \`a Faux pour ne pas l'activer.
Les valeurs 20 et 20% peuvent \^etre modif\'ees gr�ce aux variables globales au d\'ebut du
module d'\'ex\'ecution.

\textbf{La g\'eolocalisation}


\textbf{La loi de Poisson g\'eographique}



\section{La visualisation du r\'esultat}

\textbf{L'utilit\'e de la visualisation}

D\'ecrire

\textbf{La source de donn\'ees}
D\'ecrire

\textbf{L'organisation de la visualisation}
D\'ecrire

\section{Interpr\'etation de nos r\'esultats}


\section{Notre essai sur DBScan}


\section{Conclusion}
Les combinaisons int\'eressantes
Quand, comment


\section*{Acknowledgments}
Nos remerciements vont \`a Marc Plantevit pour l'enseignement de son cours "Data Mining" \`a l'Universit\'e Claude Bernard Lyon I et son accompagnement durant ce projet.


\bibliographystyle{abbrv-doi}
%\bibliographystyle{abbrv-doi-narrow}
%\bibliographystyle{abbrv-doi-hyperref}
%\bibliographystyle{abbrv-doi-hyperref-narrow}

\bibliography{dataminingbib}

\appendix

\end{document}
