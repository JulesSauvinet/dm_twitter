\documentclass[12pt]{article}
\usepackage[french]{babel}
\usepackage[utf8]{inputenc}
\usepackage{amsmath}
\usepackage{graphicx}
\usepackage[colorinlistoftodos]{todonotes}
\usepackage{float}
\usepackage{graphicx}
\usepackage{caption}
\usepackage{subcaption}
\usepackage{adjustbox}
\usepackage[euler]{textgreek}
\usepackage[left=1in,right=1in,top=1in,bottom=1in]{geometry}
%\usepackage{subfigure}
\usepackage{comment}
\usepackage{caption}
\usepackage{lastpage}
\usepackage[colorlinks,pdfpagelabels,pdfstartview = FitH,bookmarksopen = true,bookmarksnumbered = true,linkcolor = black,plainpages = false,hypertexnames = true,citecolor = black,pagebackref = true,urlcolor = black] {hyperref}
\usepackage{setspace}
\usepackage{silence}
\WarningFilter{latex}{Text page}
\usepackage{parskip}
\newcommand{\cms}{~cm\textsuperscript{-2}s\textsuperscript{-1} }

\graphicspath{{figures/}}   
% \renewcommand{\figurename}{Fig.}
\addto\captionsfrench{\renewcommand{\figurename}{Fig.}}

\begin{document}

\begin{titlepage}

\newcommand{\HRule}{\rule{\linewidth}{0.5mm}} % Defines a new command for the horizontal lines, change thickness here

\center % Center everything on the page
 
%----------------------------------------------------------------------------------------
%	HEADING SECTIONS
%----------------------------------------------------------------------------------------

\textsc{\LARGE Universit\'e Claude Bernard Lyon I}\\[1.5cm] % Name of your university/college
\textsc{\Large Master Data Science}\\[0.5cm] % Major heading such as course name
\textsc{\large Projet Data Mining}\\[0.5cm] % Minor heading such as course title

%----------------------------------------------------------------------------------------
%	TITLE SECTION
%----------------------------------------------------------------------------------------

\HRule \\[0.4cm]
{ \huge \bfseries D\'etection d'\'ev\'enements sur Twitter}\\[0.4cm] % Title of your document
\HRule \\[1.5cm]
 
%----------------------------------------------------------------------------------------
%	AUTHOR SECTION
%----------------------------------------------------------------------------------------

\begin{minipage}{0.4\textwidth}
\begin{flushleft} \large
\emph{Auteurs:}\\
Gregory \textsc{Howard} \textit{11207726} \\ 
Marine \textsc{Ruiz} \textit{11208141} \\ 
Jules \textsc{Sauvinet} \textit{p1412086}
\end{flushleft}
\end{minipage}
~
\begin{minipage}{0.4\textwidth}
\begin{flushright} \large
\emph{Professeur:} \\
Marc \textsc{Plantevit} % Supervisor's Name
\end{flushright}
\end{minipage}\\[2cm]

% If you don't want a supervisor, uncomment the two lines below and remove the section above
%\Large \emph{Author:}\\
%John \textsc{Smith}\\[3cm] % Your name

%----------------------------------------------------------------------------------------
%	DATE SECTION
%----------------------------------------------------------------------------------------

{\large \today}\\[1cm] % Date, change the \today to a set date if you want to be precise

%----------------------------------------------------------------------------------------
%	LOGO SECTION
%----------------------------------------------------------------------------------------

\includegraphics[height=3cm]{twitter}\vspace{2cm}
\includegraphics[height=3cm]{ucbl}\\
 % Include a department/university logo - this will require the graphicx package
 
%----------------------------------------------------------------------------------------

\vfill % Fill the rest of the page with whitespace

\end{titlepage}

\begin{abstract}
La d\'etection d'\'ev\`enement est un des sujets de recherche les plus importants dans l'analyse des r\'eseaux sociaux.
Les flux de donn\'ees provenant des plates-formes de r\'eseaux sociaux contiennent en g\'en\'eral beaucoup d'informations, permettant notamment de pouvoir faire de la d\'etection d'\'ev\`enement. N\'anmoins, une grande partie est bruit\'ee, notamment pas des comptes Twitter entretenus par des robots qui emp\^echent de d\'etecter des \'ev\'enements ou qui en cr\'e\'ent de faux. Il est ainsi important de comprendre comment att\'enuer l'influence du bruit pour faire de la d\'etection d'\'ev\'`enements. Notre objectif a \'et\'e de d\'etecter des \'ev\`enements \`a partir de tweets en exploitant des informations spatio-temporelles et textuelles en essayant d'\'ecarter les tweets de robots. 
\end{abstract}

%\begin{spacing}{1.15}
\pdfbookmark[1]{Contents}{toc}
\small{
\tableofcontents 
}


\newpage
% \listoffigures
% \newpage
% \listoftables 
%\end{spacing}

\section{Introduction}

\paragraph{}
Notre objectif a \'et\'e de d\'etecter des \'ev\`enements \`a partir de tweets en exploitant des informations spatio-temporelles et textuelles. Notre approche utilise la technique de d\'etection d'\'ev\'enements multi-\'echelles dans les r\'eseaux sociaux introduite dans l'article de recherche \cite{Multievents} et d\'evelopp\'e par \textsc{Ahmed Anes Bendimerad} et \textsc{Aimene Belfodil}. Nous avons adapt\'e cette version \`a nos donn\'ees \`a laquelle nous avons ajout\'e des m\'ecanismes permettant de filtrer les tweets envoy\'es par des robots en amont du clustering permettant la d\'etection d'\'ev\`enements. Nous avons \'egalement d\'evelopp\'e une approche de d\'etection spatio-temporelle d'\'ev\`enements avec l'algorithme de clustering DBScan puis nous avons compar\'e les r\'esultats des deux types de clustering.

\paragraph{}
Nous appelerons \'ev\`evenement (au sens \'ev\`evenement que nous souhaitons d\'etecter) un ph\'enom\`ene physique survenant en un point et pendant une dur\'ee bien d\'etermin\'es. Un \'ev\'enement est int\'eressant s'il suscite un nombre suffisant de tweets. Ces tweets peuvent se situer sur le lieu de l'\'ev\`enement ou bien ailleurs, notamment si celui-ci est retransmis par les m\'edias. Les tweets sur l'\'ev\`enement peuvent se produire avant, pendant et apr\`es l'\'ev\'enement. Ainsi, la d\'etection d'\'ev\`enement doit r\'eussir \`a int\'egrer ces dispersions pour localiser dans les temps et l'espace les \'ev\`enements et c'est ce qui la rend complexe.

\section{Description de nos donn\'ees}
\label{sec:desc_donnees}

\textit{Source du fichier}

\paragraph{}
Nous avons effectu\'e notre d\'etection d'\'ev\`evenent \`a partir d'un fichier de 1,7 Go comportant 10982005 tweets  r\'ecup\'er\'es sur Twitter. Ces tweets ont \'et\'e post\'e dans l'agglom\'eration newyorkaise ou par des utilisateurs newyorkais du 21 jullet 2015 au 13 novembre 2015. Chaque tweet contient les informations suivantes :
\begin{itemize}
	\item un identifiant
	\item la date et l'heure \`a laquelle le tweet a \'et\'e post\'e
	\item la position de la personne quand le tweet a \'et\'e post\'e (latitude et longitude)
	\item l'identifiant du lieu g\'eographique o\`u a \'et\'e post\'e le tweet
	\item le lieu du tweet au moment o\`u il a \'et\'e post\'e
	\item l'ensemble des r\'ef\'erences et hashtags du tweet
	\item l'identifiant de la personne qui a post\'e le tweet
	\item le pseudo de la personne qui a post\'e le tweet
	\item le nom complet de la personne qui a post\'e le tweet
	\item l'identifiant du lieu de cr\'eation du compte de la personne qui a post\'e le tweet
	\item l'information de la certification ou non du compte de l'utilisateur ayant post\'e le tweet
	\item le nombre de personnes qui suivent la personne qui a post\'e le tweet
	\item le nombre d'amis de la personne qui a post\'e le tweet
	\item le nombre de tweet d\'ej\`a post\'e par la personne qui a post\'e le tweet
\end{itemize}

Pour le d\'etection d'\'ev\`enement, nous nous sommes servi de l'identifiant du tweet, du nom de la personne qui a post\'e le
tweet, des hashtags, de la date et de la position d'envoi du tweet.

\section{Le pr\'e-traitement des donn\'ees}

\paragraph{}
Nous avons utilis\'e Knime pour effectuer un premier filtrage nous permettant d'augmenter la pertinence des \'ev\`enements d\'etect\'es. Nous supprimons les 0.5\% d'utilisateur qui tweetent les plus. Cela nous permet d'homog\'en\'eiser nos donn\'ees. 
D'une part, il y a une probabilit\'e importante qu'un utilisateur qui poste une tr\'es forte quantit\'e de tweet soit un robot. D'autre part, nous nous sommes dit qu'un utilisateur r\'eel ayant une activit\'e sur Twitter bien sup\'erieure \`a celle des autres utilisateur avait en moyenne moins de contenu descriptif d'\'ev\`evenement dans ses tweets et plus de "tweets bruits", c'est \`a dire des tweets nous donnant aucune informations sur le d\'eroulement d'un quelconque \'ev\`evenement (e.g "\#lol", ou encore "\#oklm avec mon refr\'e"). Nous avons ensuite confirm\'e cette intuition empiriquement en observant notre jeu de donn\'ees. Cette technique \'etant tr\`es appromative, nous ne l'utilisons que pour filtrer une quantit\'e peu importante d'utilisateurs.
\newline
Nous avons joint le workflow Knime robotFilter.knwf permettant de faire ce filtrage \`a l'archive de rendu du projet.

\paragraph{}
La dimension g\'eographique \'etant cruciale dans notre m\'ethode de clustering, nous avons \'egalement filtr\'e les tweets ne poss\'edant pas de position g\'eographique sur le lieu de leur envoi.
\newline
Nous n'avons \'a la fin plus que 1,600,000 de tweets et un fichier de 250Mo. 


\section{L'approche globale}

\paragraph{}
Pour notre premier algorithme, nous faisons des sous-ensembles de tweets en d\'ecoupant
nos donn\'ees par date. On a ainsi un ensemble de tweets par jour. On applique sur chaque
ensemble journalier de tweets, l'algorithme de d\'etection d'\'ev\`enement multi-\'echelle.

\paragraph{}
Pour notre second algorithme, nous supprimons avant tout traitement, les comptes robots
que nous sommes susceptibles d'avoir dans nos donn\'ees. Il est fort probable qu'ils
faussent notre clustering.

\paragraph{}
De plus, chacun de nos \'ev\'enements est d\'ecrit par 20 hashtags pertinents. Ces hashtags
pertinents sont d\'efinis comme \'etant les plus r\'ecurrents dans les tweets contenus dans
l'\'ev\'enement. Cependant, certains hashtags sont pr\'esents dans la description de
beaucoup d'autres \'ev\'enements, ce qui les rend moins pertinents. Nous voulons les
d\'etecter et ne pas les prendre en compte lors du clustering afin d'avoir une vraie description
de l'\'ev\'enement.

\newpage


\section{La d\'etection d'\'ev\`enement multi-\'chelle}

\paragraph{}
La technique de d\'etection d'\'evenement n'est autre que celle \'evoqu\'ee dans l'article \cite{Multievents}.

L'algorithme calcule la matrice de similarit\'e et construit les clusters.

MOI
\newline 
MOI 
\newline
MOI

\newpage

\section{Adaptation de l'algorithme de d\'etection multi-\'echelle et int\'egration du filtrage des robots}

\subsection{Description des \'el\'ements utiles \`a la d\'etection d'\'ev\'enements}

\textbf{Le tweet}
\newline
Un tweet est d\'ecrit par : 

\begin{itemize}
\item son identifiant
\item le nom du Tweetos
\item les hashtags
\item la date et l'heure \`a laquelle le Tweetos a tweet\'e
\item sa position en latitude/longitude au moment du tweet
\item La date
\end{itemize}

Notre jeu de donn\'ees de tweets s'\'etale sur 103 jours. Puisque nous voulons faire de la d\'etection d'\'ev`enements (ponctuels) localis\'es -entre autres- dans le temps, nous avons d\'ecoup'e notre jeu de donn\'ee initial en paquets jour par jour qui contiennent chacun plus de 15000 tweets.

\textbf{La matrice de similarit\'e}
\newline
La matrice de similarit\'e est une matrice triangulaire sup\'erieure de taille (nombre de
tweets) * (nombre de tweets).
D\'ecrire construction
Citer l'article en parlant rapidement de la compression de Haar

\textbf{Le clustering}
\newline
Le clustering se fait \`a partir de la matrice de similarit\'e.
D\'ecrire mieux l'algo
\newline
Citer le jar et expliquer rapidement ce qu'on maximise / minimise dans l'algorithme

\textbf{L'\'ev\'enement}
\newline
On d\'ecrit un \'ev\'enement par :
\begin{itemize}
\item une heure de d\'ebut et une heure de fin qui permettent de calculer une dur\'ee
\item une position centrale (longitude, latitude) calcul\'ee \`a partir de la moyenne de toutes
les positions (moyenne des longitudes, moyenne des longitudes)
\item le nombre de personnes diff\'erentes ayant post\'e des tweets
\item la liste des 20 hashtags les plus importants permettant de le d\'ecrire
\end{itemize}



\textbf{a placer ou il faut}
\newline
on traite ensuite tous les tweets d'une m\^eme journ\'ee
\newline
on enl\`eve ceux qui sont r\'egi par une loi G\'eom\'etrique (a voir)
\newline
on fait du clustering sur les tweets et on renvoie les clusters pertinents sous forme
d'\'ev\'enements

\section{Les options sur le clustering}

\textbf{L'\'elasticit\'e}

\paragraph{}
On suppose qu'un hashtag est fr\'equent pour un ensemble de tweets donn\'e s'il apparait
plus de 20 fois.
\newline
Ce nombre est statique, ce qui ne permettrait pas de d\'etecter de petits \'ev\'enements dans
une journ\'ee o\`u il y a eu peu de tweets. A l'inverse, on d\'etecte beaucoup d'\'ev\'enements sur
une journ\'ee o\`u il y a eu beaucoup de tweets.
Cela peut \^etre int\'eressant, mais il est parfois plus judicieux de d'adapter au nombre de
tweets post\'es dans une m\^eme journ\'ee. Ainsi, on d\'efinit un hashtag fr\'equent s'il apparait
dans plus de 20% des tweets.
Pour coder cette option nous avons d\'efinit un bool\'een, que l'on met \`a Vrai pour activer
l'\'elasticit\'e et \`a Faux pour ne pas l'activer.
Les valeurs 20 et 20% peuvent \^etre modif\'ees gr�ce aux variables globales au d\'ebut du
module d'\'ex\'ecution.

\textbf{La g\'eolocalisation}


\textbf{La loi de Poisson g\'eographique}


\section{La d\'etection d\'ev\`enement avec DBScan}


\section{R\'esultats}



\section{Conclusion}
Les combinaisons int\'eressantes
\newline
Quand, comment


\section*{Acknowledgments}
Nos remerciements vont \`a Marc Plantevit pour l'enseignement de son cours "Data Mining" \`a l'Universit\'e Claude Bernard Lyon I et son accompagnement durant ce projet.



\bibliographystyle{abbrv}
\bibliography{dataminingbib}

\appendix

\end{document}














